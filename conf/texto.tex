% Método de entrada de texto (no código fonte):

\usepackage[utf8]{inputenc}

% Idiomas utilizados (o idioma principal vem por último):

\usepackage[english,portuges]{babel}

% Método de saída de texto (impressão em PDF):

\usepackage[T1]{fontenc}

% Geometria da página:

\usepackage[hmarginratio=1:1,top=3cm,bottom=3cm,left=2cm,right=2cm,columnsep=20pt]{geometry}

% Para recuo no primeiro parágrafo (no idioma inglês, não é costume recuar):

\usepackage{indentfirst}

% Microtipografia:

\usepackage{microtype} 

% Cores extras:

\usepackage{xcolor}

\definecolor{verde}{rgb}{0.2, 0.50, 0.25}

% Para inclusão de figuras:

\usepackage{graphics}
\usepackage{graphicx}

% Para mesclar linhas e colunas em tabelas:

\usepackage{multirow}
\usepackage{multicol}

% Para controlar posição de elementos flutuantes (gráficos, tabelas etc.):

\usepackage{float}

% Distância relativa entre linhas:

\linespread{1.05} 

% Para formatação de autores e instituições:

\usepackage{authblk}
\usepackage[pdfborder={0 0 0}]{hyperref}% For email addresses

% Diversos:

\usepackage[hang, small,labelfont=bf,up,textfont=it,up]{caption} 
\usepackage{booktabs}
\usepackage{paralist} 
\usepackage{abstract} 
\usepackage{titlesec} 
\usepackage{pslatex}

% Comandos de palavras-chave:

\providecommand{\palavraschave}[1]{\textbf{\indent\textit{Palavras-chave:}} #1} % https://tex.stackexchange.com/a/137307/91816
\providecommand{\keywords}[1]{\textbf{\indent\textit{Keywords:}} #1}

% Formatação referente ao corpo do texto e cabeçalho.

\renewcommand{\abstractnamefont}{\normalfont\bfseries} 
\renewcommand{\abstracttextfont}{\normalfont\small\itshape} 
\renewcommand\thesection{\Roman{section}} 
\renewcommand\thesubsection{\Roman{subsection}} 
\titleformat{\section}[block]{\large\scshape\centering}{\thesection.}{1em}{} 
\titleformat{\subsection}[block]{\large}{\thesubsection.}{1em}{}
% Modelo de relatório de experimento em formato de artigo científico

% Criado a partir de "Modelo de Relatório abnTeX", de autoria de Gustavo Yudi Bientinezi Matsuzake, disponibilizado na Galeria Overleaf (https://pt.overleaf.com/latex/templates/modelo-de-relatorio-abntex/cffmvqrhsprn).

% Definição do documento e pacotes utilizados. 

\documentclass[12pt,a4paper,twoside]{article}

% Configurações:

\input{conf/texto.tex}
\input{conf/matematica.tex}
\input{conf/ciencia.tex}
\input{conf/ambientes.tex}

% Para citações:

\usepackage[brazilian,hyperpageref]{backref}
\usepackage[alf]{abntex2cite}

% EDITE A PARTIR DAQUI:

\def \tituloportugues {COLOQUE SEU TÍTULO EM PORTUGUÊS AQUI, ELE PASSARÁ PARA A OUTRA LINHA AUTOMATICAMENTE SE PRECISO}  % Título. [EDITAR]
\def \subtitulo {COLOQUE O SUBTÍTULO EM INGLÊS AQUI, ELE PASSARÁ PARA A OUTRA LINHA AUTOMATICAMENTE SE PRECISO}  % Título em inglês. [EDITAR]
\setcounter{page}{1} % página inicial do relatório
\def \nomeedital {PIBIC / DGP}
\def \nomerelatorio {Relatório Final de Iniciciação Científica}
\def \tema {Física} % pode mudar para "Eletromagnetismo", "Física Quântica" etc.
\def \coordenadas {\textit{Grupo de Fluidos Complexos}, Brasília, 2019.} % Título em inglês. [EDITAR]
\def \paginainternet {https://http://proic.unb.br/}
\definecolor{cortextoquadro}{RGB}{0,58,112} % Cor do texto dos quadros do leiaute.
%\definecolor{cortextoquadro}{RGB}{0,132,61}
\definecolor{corfundoquadro}{RGB}{200,200,200} % Cor de fundo dos quadros do leiaute.
\def \fontetexto {\bfseries} % Configurações de fonte para textos dentro dos quadros do leiaute.
\def \logorevista {figs/logo-unb.png} % Defina o arquivo do logotipo.
\def \larguralogo {2.3cm} % Defina a largura com que o logotipo aparecerá na capa.

% As configurações de layout estão no arquivo mostrado abaixo:
\input{conf/layout.tex} % Não mexa nesta linha!

% Coloque abaixo os nomes dos autores e o número de suas instituições:

\author[1]{Coloque Nome do Autor Um}
\author[2]{Coloque Nome do Autor Dois}
\author[1]{Coloque Nome do Autor Três}
\author[2]{Coloque Nome do Autor Quatro}

% Coloque abaixo os números e nomes das instituições:

\affil[1]{Coloque aqui o nome da instituição 1.}
\affil[2]{Coloque aqui o nome da instituição 2.}
\affil[3]{Coloque aqui o nome da instituição 3.}

% Ajustes diversos antes de começar o documento (últimos preparativos),
% tais como atribuição da variável "title", configurações para resolver erros de compilação etc.

\input{conf/ultimos-preparativos.tex}

%-----------------------------

% Aqui damos início ao documento.

\begin{document}

\maketitle

\thispagestyle{capa}

% Resumo e palavras-chave. 

\hrule

\selectlanguage{portuges}
\begin{abstract}
\noindent
\input{texto/resumo.tex} % [EDITAR] %
\vspace{2\baselineskip}
\palavraschave{iniciação científica, física.} % [EDITAR] %
\end{abstract}
\vspace{5mm}

\hrule

\selectlanguage{portuges}

\section{Introdução} % [EDITAR]

    \input{texto/introducao.tex} % [EDITAR]

\section{Metodologia} % [EDITAR]

    \input{texto/metodologia.tex} % [EDITAR]

\section{Resultados} % [EDITAR]

    \input{texto/resultados.tex}

\section{Discussao/Conclusão} % [EDITAR]

    \input{texto/discussao.tex} % [EDITAR]

\bibliography{relatorio}

\end{document}

% Tutoriais LaTeX:

% <http://www.mat.ufmg.br/~regi/topicos/intlat.pdf>
% <http://ece.uprm.edu/~caceros/latex/introduction.pdf>

%-----------------------------

% LaTeX Online (free account):

% <https://www.overleaf.com/>

%-----------------------------

% Códigos matemáticos LaTeX:

% <http://www.codecogs.com/latex/eqneditor.php>

%-----------------------------

% Construir tabelas no LaTeX:

% <http://www.tablesgenerator.com/#>

%-----------------------------